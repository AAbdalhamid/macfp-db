% !TEX root = macfp_2017_gasphase.tex

\subsection{Future Plans} \label{sec:CPS_session_3}

Future steps include the development of a digital archive dedicated to the condensed phase subgroup, possibly using the same platform as the gas phase subgroup~\cite{MaCFP_repository}.  In parallel, the standards for the experimental data sets will be established.  The following standards are proposed:

\begin{itemize}
\item Each studied material must have clearly defined chemical composition.  The material's physical attributes, such as color, initial density and thickness, geometry of reinforcement (in the case of structural composites), must be provided.  The material should be readily available, preferably, from multiple distributors.
\item The material should be conditioned prior to all experiments in a well-defined atmosphere with these conditions specified.  For hydrophilic materials, the initial moisture content should be reported.
\item The experiments used to determine properties may consist of milligram-scale and/or gram-scale tests.  Milligram-scale tests (such as TGA) are expected to be conducted under thermally thin conditions, $i.e.$ conditions for which the sample temperature is spatially uniform and resolved in time.  Gram-scale tests (such as FPA gasification experiments~\cite{Chaos:2011}) are expected to be conducted under non-thermally thin conditions, $i.e.$ conditions for which transport properties have significant impact on the measured quantities.  Gram-scale tests must have well-defined thermal boundary conditions, more specifically, heat fluxes incident on all sample surfaces should be specified as a function of surface temperature.  If the material sample is mounted onto thermal insulation, the properties of this insulation must be provided.  The composition of the gaseous environment inside all test apparatus must be defined. For all tests, the size and mass of the samples must be fully specified.
\item Each data set may contain either milligram- and gram-scale test results or, alternatively, only gram-scale test results.  In the latter case, the results from multiple experiments performed at a range of heating conditions must be reported and include time-resolved sample mass as well as sample temperature measurements (at the surface and/or at an in-depth location). The conditions of the tests (for example heating rate, temperature range, percentage of oxygen in the atmosphere) must be defined.
\item Heat of combustions of gaseous pyrolyzate produced by the material must be measured using a Cone Calorimeter, FPA, or Microscale Combustion Calorimeter.
\item Additional data, including chemical composition of the gaseous pyrolyzate, and thermal conductivity, emissivity, radiation absorption coefficient and mass diffusivity of the solid, are desirable but not required.
\item All experimental data must contain information on their uncertainties.
\end{itemize}

Multiple experimental data sets for the same material will be allowed into the repository, provided that each of them satisfies all established requirements.  One key requirement for each pyrolysis property set, which will be generated from the experimental datasets, is a demonstration of how these property values capture all data in at least one experimental dataset, $e.g.$ if the dataset contains the results of controlled-atmosphere cone calorimetry experiments and TGA, the developer of the pyrolysis property set will be required to produce predictions of both of these experiments and provide input files that were used to generate these predictions.  Quantitative criteria will be developed to characterize the quality of each prediction. 

It is proposed that, initially, experimental data sets will be developed for relatively simple materials that are isotropic in nature and do not exhibit complex mechanical behavior such as melt flow, delamination or intumescence.  Examples of such materials include cast poly(methyl methacrylate) and high-impact polystyrene.  Demonstrating that the pyrolysis property sets can be used to successfully predict compartment-scale fire growth will be a longer term goal of this effort.  One potential target geometry for the full-scale experiments is upward and lateral flame spread in a flammable corner, which is realized in several flammability standards~\cite{NFPA,EN,ISO}.  It is proposed that well-instrumented versions of these standard experiments be carried out to serve as a modeling target for comprehensive gas and condensed phase models of fire growth.

In closing, the co-chairs of the condensed phase subgroup of the MaCFP Working Group have now started discussions for the organization of a second workshop. Interested individuals/organizations are encouraged to contact the co-chairs~\cite{MaCFP_website} in order to participate in preparations for the workshop and in the construction of the digital archive described above.

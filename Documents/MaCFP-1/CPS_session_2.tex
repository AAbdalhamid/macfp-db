% !TEX root = macfp_2017_gasphase.tex

\subsection{Summary of the Planning Meeting} \label{sec:CPS_session_2}

As explained in section~\ref{sec:intro}, in addition to being a first technical meeting for the gas phase subgroup, the June-2017 MaCFP workshop served as a planning meeting for the condensed phase subgroup. The planning meeting featured an introductory presentation by the co-chairs of the condensed phase subgroup, followed by seven invited presentations and two periods for an open discussion. Hard copies of the presentations can be found in~\cite{MaCFP_wks_presentations}.

The introductory presentation~\cite{CPS:Bruns} presented an overview of the motivation, purpose, and goals of the condensed phase subgroup.  It was emphasized that fire phenomena can only be predicted with robust coupling between condensed and gas phase models.  Consequently, it will be necessary for the two subgroups of MaCFP to work closely in the planning and analysis of validation data as well as in subsequent model development.  Several of the challenges associated with condensed phase fire physics were mentioned.  Overcoming these challenges requires systematic verification and validation of condensed phase models.  Several concepts from validation and verification were reviewed including the so-called ``validation pyramid" as a heuristic for systematically validating complex models via sequential validation of various submodels.  The International Workshop on Measurement and Computation of Turbulent Nonpremixed Flames (known as the TNF workshop) was mentioned as a model for organizing the condensed phase subgroup's activities.  The reference to the TNF model led to a brief description of a proposed plan for the subgroup's work as presented in the White Paper~\cite{MaCFP_website} prepared by the co-chairs prior to the workshop.  The core of the proposal is to facilitate communication between experimentalists and modelers by providing web-based management of four elements:  (1) experimental data; (2) numerical models; (3) parameter sets and associated comparisons between model predictions and experimental data; and (4) a discussion forum.  For each of these elements, the presentation provided a brief explanation as well as some proposed constraints.  First, the experimental data should initially focus on scenarios in which flaming is not present so that condensed phase physics may be isolated.  This data will need to follow some requirements for formatting and review.  Second, the numerical models should be open source, well-documented, and include at least heat transfer and decomposition reaction kinetics.  Third, the parameter sets and comparisons should be complete and the link between the underlying data and the parameter values should be specified.  The Fire Dynamics Simulator (FDS) Validation Guide~\cite{FDS_Validation_Guide} was mentioned as a good example of such comparisons.  Finally, several topics such as missing experimental data, needed model developments, and computational challenges were suggested for the discussion forum.  It was noted that a successful discussion forum will require sustained community participation.

The first invited presentation~\cite{CPS:Leventon} began with an overview of different condensed phase models, from early heat-transfer-based analytical models for ignition up to modern computational pyrolysis solvers such as FDS~\cite{FDS_Math_Guide}, Gpyro~\cite{Lautenberger:2014} and ThermaKin~\cite{Stoliarov:2014}.  These modern computational models rely on a relatively large number of material properties used to characterize pyrolysis behavior.  A list of common material properties used in computational pyrolysis models is provided in Table~\ref{tab:table1}.  Identifying values for these many parameters presents a challenge especially as the values can change significantly as a material heats and decomposes.  The remainder of the presentation focused on describing a procedure for determining the kinetic and thermodynamic properties of materials developed at the University of Maryland~\cite{Stoliarov:2015}.  This procedure relies on data from three milligram-scale experiments:  (1) thermogravimetric analysis (TGA) for decomposition reaction kinetics; (2) differential scanning calorimetry (DSC) for heat capacities and heats of decomposition reactions; and (3) microscale combustion calorimetry (MCC) for heats of combustion of gaseous pyrolyzates.  The presentation described these experiments and procedures for extracting material properties from the appropriate data.  Throughout the discussion, data for poly(butylene terephthalate) (PBT) was used as an example. 

\begin{table}
\caption{Material properties required by state-of-the-art computational pyrolysis models.}
\centering
\footnotesize
\makebox[\textwidth]{
\begin{tabular}{p{1.75in}p{1.95in}p{1.75in}}
\toprule
 Kinetic & Thermodynamic & Transport \\
\midrule
\midrule
 Pre-exponential factors     & Specific heat capacities                 & Thermal conductivities \\
 Activation energies            & Heats of decomposition reactions & Emissivities \\
 Stoichiometric coefficients & Heats of combustion                     & Absorption coefficients \\
                                            &                                                      & Mass diffusivities \\
\bottomrule
\end{tabular}
}
\label{tab:table1}
\end{table}

The second invited presentation~\cite{CPS:Brown} discussed current work on validating models of solid reacting materials.  Surface temperature measurements using thermophosphors are being explored, and datasets for solid reactive materials are being generated.  The focus at Sandia National Laboratories (Sandia) is on the high heat flux regime.  A number of test facilities are available for high heat flux ignition experiments including the Sandia solar furnace which can provide a heat flux of 5 MW/m$^2$.  In addition to experimental work, Sandia is developing a code for fire modeling (Fuego~\cite{SIERRA/Fuego}) and a code for reacting solid materials (Aria).

The relationship between gas and condensed phase physics was discussed in the third invited presentation~\cite{CPS:Richard}.  The large number of physical processes occurring at the solid-gas interface were enumerated.  It was emphasized that ignition and flame spread are significantly influenced by the details of transport and chemistry occurring at the interface.  A review of boundary layer theory was provided as well as a discussion of the reacting boundary layer theory of Emmons.  The presentation concluded by highlighting the need to develop better models of turbulence, heat transfer, and combustion in the near-wall region of the boundary layer to account for chemical and blowing effects corresponding to pyrolysis.

The fourth invited presentation~\cite{CPS:Hostikka} provided a discussion of the solid model implemented in FDS~\cite{FDS}.  The presentation began by presenting a schematic of the physical processes involved in burning materials and emphasized the multi-scale nature of the problem.  The governing equations for condensed phase species and energy conservation as well as the pore gas conservation equations for species, energy and momentum, were presented.  The need to limit the model to include only the important physics was noted.  Following on from this point, the presentation listed the major assumptions made by the FDS solid model.  Specifically, the FDS solid model assumes one-dimensional transport, no mass accumulation (and therefore instantaneous mass transfer), thermal equilibrium between gases and solids, and assumes that a heat of reaction may be used to account for the energy contribution of the decomposition reactions.  The FDS solid model is regularly verified (in terms of heat conduction, radiation, mass conservation, and reaction rate) and validated (in terms of mass loss rate and heat release rate for burning polymer slabs).  Several special topics for future pyrolysis model development were mentioned including spectral radiation, shrinking and swelling, pressure build-up, multi-dimensional effects, and solid mechanical considerations associated with fracturing of char.  The presentation concluded by suggesting that these special topics might be appropriate for further exploration by the MaCFP condensed phase subgroup.

The challenge of coupling condensed and solid phase models was explored in the fifth invited presentation~\cite{CPS:Wang}.  In contrast to the multi-experiment approach developed at the University of Maryland~\cite{CPS:Leventon}, FM Global calibrates all material properties using data from a single experiment, namely the fire propagation apparatus (FPA).  A one-dimensional model with a single-step Arrhenius reaction is then fit to the FPA data using optimization with the shuffled complex evolution (SCE) algorithm.  The resultant pyrolysis properties are then used as inputs in a CFD fire model (FireFOAM~\cite{FireFOAM}) to simulate full-scale fire scenarios.  Validation of this approach has been performed for several additional FPA scenarios.  An important application of fire models for FM Global is understanding fire spread in warehouse rack storage.  FireFOAM has been used to predict heat release rate in 3-tier, 5-tier, and 7-tier rack storage of cardboard boxes using properties obtained by the FPA/SCE material property calibration procedure.  The presentation also discussed applications involving boxes of plastic cups and large rolls of paper.  The paper rolls present a unique challenge in that delamination of outer layers of paper had to be accounted for.  Several lessons were provided in conclusion.  First, coupling of gas and condensed phase models is necessary for real-world problems, and the appropriate level of model complexity is determined by the problem.  Second, validation needs to occur both for the decoupled and coupled models at multiple scales.

The sixth invited presentation~\cite{CPS:Rein} discussed recent work on assessing the appropriate level of model complexity.  Beginning with a clear statement of the goal to ``up-scale" from fundamental physics and chemistry to real fire behavior, the presentation laid out some of the many challenges in the path of achieving that goal.  One of those challenges is choosing the appropriate level of model complexity.  The number of parameters in pyrolysis models can vary from just a few to over 30 for some of the more detailed models in existence.  The problem of complexity is one of finding the minimum number of parameters required to attain an acceptable level of error.  In the recent work presented in Refs.~\cite{Rein:2013,Rein:2015}, this problem has been addressed by systematically decreasing model complexity used to predict the pyrolysis of a vertical slab of PMMA exposed to varying levels of heat flux and oxygen.  It was found that it is not helpful to increase the complexity of the chemical model unless a sufficiently complex model of heat transfer is used.  Furthermore, additional complexity corresponds to increased uncertainty, and so complex models should only be used in the presence of sufficient, high quality data.

Finally, the seventh invited presentation~\cite{CPS:Lautenberger} provided an overview of pyrolysis modeling with Gpyro.  Gpyro is an open-source three-dimensional pyrolysis model with user-specified complexity.  Additionally, Gpyro may be coupled to FDS with some limitations ($e.g.$ Cartesian geometries, no shrinkage or swelling, and no burn-away).  Current work involves coupling to ABAQUS for mechanical calculations.  A critical part of any pyrolysis solver is the material property models allowed.  Gpyro treats material properties as weighted sums of species properties with a power law dependence on temperature.  Additionally, permeability and thermal conductivity may be anisotropic which can be important for materials such as wood.  After going through the form of the conservation equations, the presentation provided some details on the numerical schemes employed by Gpyro.  The time-stepping is fully implicit to ensure stability of the solution, and an alternating direction tri-diagonal matrix algorithm is utilized for speed.  Several verification cases have been carried out including for a sphere with internal heat generation.  As an example of the power of detailed pyrolysis modeling, the presentation gave the example of wood pyrolysis at both small and large external heat fluxes, with the ``fast" case producing significantly more tar as compared to the ``slow" case.

The invited presentations served to establish a foundation for the two periods of open discussion that were scheduled in the workshop.  These open discussion periods were crucial for fielding input from the research community at large.  The issue of heating rate in small scale experiments was discussed.  Some believe that the heating rates used in small-scale tests should emulate realistic fire heating rates, but it was noted that chemical reaction kinetics generally depend on temperature (not heating rate) and increasing the heating rate does not significantly change the temperature range over which solid decomposition reactions take place.  Furthermore, high heating rates can lead to temperature and species concentration gradients which prevent meaningful interpretation of results of small-scale tests such as TGA.  Several participants suggested that TGA should be coupled with gas analysis (e.g., FTIR) as this information could be important for the gas phase physics.  Similarly, the impact of oxygen concentration should be explored further in order to model the transitional regimes of ignition, spread, and extinction in contrast to steady burning.  For all small-scale tests, it was suggested that it should be necessary to precisely describe the range of validity of the parameters as a consequence of how they are determined.  Much of the open discussion time was devoted to identifying appropriate validation data sets.  Some participants suggested that it is important to have large-scale data (such as the FM Global parallel panel test or the standard room corner tests) early on in order to better guide subsequent model development and experimentation.  This would not negate the necessity of small-scale tests for model parameterization or validation of sub-models, but inversely, this illustrates the pertinence of the up-scaling approach. Another issue that arose is the appropriate selection of test materials.  A balance should be struck between simplicity for modeling purposes and real-world application.



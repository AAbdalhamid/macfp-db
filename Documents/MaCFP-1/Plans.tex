% !TEX root = macfp_2017_gasphase.tex

\subsection{Current and Future Plans} \label{sec:plans}

The gas phase session of the June-2017 MaCFP workshop provided a first opportunity to demonstrate the benefits and potential impact of activities organized by the IAFSS MaCFP Working Group. The session provided a community-wide forum for in-depth technical discussions of a first suite of experimental-computational comparisons corresponding to an initial list of target experiments. The session was well attended (with 120 registered participants) and the first general lesson from the workshop is that MaCFP successfully responds to a need for greater levels of integration and coordination in fire research. The fire science community is small, fragmented and geographically dispersed: MaCFP is an effort to meet the resulting organizational challenge, to build an international collaborative framework, and to provide a critical mass of researchers for topics central to the development of a fundamental understanding of fire phenomena. While MaCFP is currently focused on building a collaborative framework between computational and experimental researchers around the topic of the validation of CFD-based fire models, we envision that MaCFP can be extended to incorporate efforts focused on other topics, or can be emulated and inspire other efforts.

The gas phase session of the first MaCFP workshop also led to a number of technical lessons and outcomes that will help shape the future activities of MaCFP. First, as discussed in section~\ref{sec:cfd_review}, the performance of CFD-based fire models depends on both the quality of the computational grid (and in particular its ability to resolve the dynamically-controlling length scales of the simulated problem) and the accuracy of the physical models (used to describe subgrid-scale turbulence, combustion, radiation, {\it etc}). In this context, it is important to emphasize that the submissions made by the seven different computational groups represented at the MaCFP workshop correspond to fine grid resolution at the millimeter- or centimeter-scale. Under high-resolution simulation conditions, the impact of numerical errors is reduced and many of the discrepancies between experimental data and computational results may be attributed to modeling errors. While fine-grained simulations are considered as a necessary step, and provide valuable insights into the accuracy of physical models, they are not representative of engineering-level simulations that typically use coarser grids: there is therefore an unmet need for MaCFP to also evaluate physical models in coarse-grained simulations that are more representative of the CFD practice. This will be addressed in future editions of the MaCFP workshop series. Note that one objective of the MaCFP Working Group is to develop guidelines for CFD practitioners for the design of the computational grid as well as reference material on the domain of validity of the different physical models available in current CFD-based fire models.

Furthermore, discussions of the different target experiments revealed a number of limitations in available experimental databases that are worth pointing out for future studies: (1) the databases are often limited to small-scale, weakly-to-moderately turbulent flames and there is a need to provide more data for large-scale fully-developed turbulent flames; (2) the databases are often limited to measuring temporal means and there is a need to provide data on fluctuation magnitudes; (3) the databases are often limited to the flame near-field and there is a need to provide data over the full flame region; and (4) the databases are often focused on characterizing the flow field or the temperature field, but not both, and there is a need to provide more comprehensive data sets including flow velocities, temperatures, and also soot volume fractions, radiation intensities and heat fluxes to surfaces. Note that the availability of quality data on radiation intensities and heat fluxes to surfaces is a requirement for future progress on simulations of flame spread phenomena. We hope that the wish list above will inspire a new generation of experimentalists and motivate new experimental studies.

Finally, as discussed in section~\ref{sec:intro}, the initial list of target experiments selected for the first MaCFP workshop had a limited scope corresponding to (mostly) non-sooting or only weakly-sooting flames, supplied with gaseous or liquid fuel, and without compartment effects. There is now a need to extend the scope of MaCFP to include target experiments that bring detailed information on a number of key fire processes, for instance: thermal radiation, soot formation and oxidation, flame spread along solid flammable materials (see the next section), and also ignition phenomena and compartment effects. In addition, the application of current fire models to the simulation of water-based fire suppression systems ($i.e.$ sprinkler or mist systems) requires additional experimental data and validation tests on water droplet dispersion, evaporation and radiation blockage. The intent of MaCFP is to expand the list of target experiments. It is also to keep re-visiting the initial list for further insights into basic flow and combustion phenomena and for additional tests with coarse-grained simulations. 

In closing, the organizing committee of the MaCFP Working Group has now started preliminary discussions for the organization of a second workshop. Interested individuals/organizations are encouraged to contact the committee~\cite{MaCFP_website} in order to join MaCFP, influence the selection of new target experiments, participate in discussions on the structure, format and scope of MaCFP, participate in discussions on the location and time of the second workshop. Note that the experimental and computational databases corresponding to Cases 1-5 are hosted on the MaCFP repository~\cite{MaCFP_repository} and are available to the fire research community as reference data for future experimental and/or computational studies. The MaCFP Working Group is committed to continuously update the repository.
